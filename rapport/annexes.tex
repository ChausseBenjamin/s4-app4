\newpage
\appendix
\section{Shématiques}
\label{appdx:schematics}


\quickfig{Additionneur 1 Bit}{add1b}{.9}{assets/add1b.png}
% \todo{Insérer le schéma de l’additionneur à 3 bits.}
\quickfig{Additionneur 3 Bits}{add3b}{.8}{assets/add3b.png}

% \todo{Insérer le schéma de la porte AND à 3 bits.}
\quickfig{ET 3 Bits}{and3b}{.8}{assets/and3b.png}

\quickfig{Multiplexeur de 2 entrées de 1 Bits}{mux2x1b}{1}{assets/mux2x1b.png}
\quickfig{Multiplexeur de 2 entrées de 3 Bits}{mux2x3b}{.8}{assets/mux2x3b.png}

\todo{Insérer le schéma du tampon de sortie.}


% \todo{Insérer le schéma de l’UAL.}
\quickfig{Unité d'arithmétique logique (UAL)}{alu}{1}{assets/alu.png}


\begin{align}
  Y &= \overline{(T_2T_3)} = o_{ut}  \\
  &= \overline{(\overline{[aT_1]}\cdot\overline{[bs]} )} \\
  &= \overline{(\overline{[a\overline{s}]}\cdot\overline{[bs]} )} \\
  &= \overline{\overline{[a\overline{s}]}} + \overline{\overline{[bs]}} \\
  Y &= a\overline{s} + bs \\
  \overline{Y} &= \overline{(a\overline{s} + bs)} \\
  \overline{Y} &= \overline{(a\overline{s})} \cdot \overline{(bs)} \\
  \overline{Y} &= (\overline{a}+\overline{\overline{s}}) \cdot (\overline{b}+\overline{s}) \\
  \overline{Y} &= (\overline{a}+s) \cdot (\overline{b}+\overline{s}) \\
\end{align}

\quicktab{Additionneur 1 Bit}{add1b}{cccccccc}{
  \toprule
  \multirow{2}{*}{$c_{in}$} &
  \multirow{2}{*}{$a$} &
  \multirow{2}{*}{$b$} &
  $T_1$     &
  $D$       &
  $T_2$     &
  $s_{um}$  &
  $c_{out}$ \\
  & & &
  $\overline{(ab)}$ &
  $a\oplus b$ &
  $\overline{(Dc_{in})}$ &
  $D\oplus c_{in}$ &
  $\overline{(T_1T_2)}$
  \\
  0&0&0&1&0&1&0&0 \\
  0&0&1&1&1&1&1&0 \\
  0&1&0&1&1&1&1&0 \\
  0&1&1&0&0&1&0&1 \\
  1&0&0&1&0&1&1&0 \\
  1&0&1&1&1&0&0&1 \\
  1&1&0&1&1&0&0&1 \\
  1&1&1&0&0&1&1&1 \\
  \bottomrule
}

\todo{Ajouter la table de vérité, la fonction $Y$ (PUN) et la fonction $\overline{Y}$ (PDN).}
