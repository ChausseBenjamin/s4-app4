\documentclass[a11paper]{article}

\usepackage{karnaugh-map}
\usepackage{subcaption}
\usepackage{tabularx}
\usepackage{titlepage}
\usepackage{document}
\usepackage{booktabs}
\usepackage{multicol}
\usepackage{float}
\usepackage{varwidth}
\usepackage{graphicx}
\usepackage{siunitx}
% \usepackage[toc,page]{appendix}
\usepackage[usenames,dvipsnames]{xcolor}

\title{Rapport d'APP}

\class{Logique Combinatoire}
\classnb{GEN420 \& GEN430}

\teacher{Keven Deslandes}

\author{
  \addtolength{\tabcolsep}{-0.4em}
  \begin{tabular}{rcl} % Ajouter des auteurs au besoin
      Benjamin Chausse & -- & CHAB1704 \\
      Shawn Couture    & -- & COUS1912 \\
  \end{tabular}
}

\newcommand{\todo}[1]{\begin{color}{Red}\textbf{TODO:} #1\end{color}}
\newcommand{\note}[1]{\begin{color}{Orange}\textbf{NOTE:} #1\end{color}}
\newcommand{\fixme}[1]{\begin{color}{Fuchsia}\textbf{FIXME:} #1\end{color}}
\newcommand{\question}[1]{\begin{color}{ForestGreen}\textbf{QUESTION:} #1\end{color}}

\newcommand{\quicktab}[4]{
  \begin{table}[H]
    \centering
    \caption{#1}
    \label{tab:#2}
    \begin{tabular}{#3}
      #3
    \end{tabular}
  \end{table}
}

\newcommand{\quickfig}[4]{
\begin{figure}[H]
  \centering
  \includegraphics[width=#3\textwidth]{#4}
  \caption{#1}
  \label{fig:#2}
\end{figure}
}

\begin{document}
\maketitle
\newpage
\tableofcontents
\newpage

\section{Plan de vérification}
\todo{Rédiger le plan de vérification.}

\section{Conception de l'UAL}

\subsection{Additionneur à 3 bits}
Voir schéma en Annexe~\ref{appdx:schematics}.
\todo{Expliquer brièvement.}

\subsection{Porte AND à 3 bits}
Voir schéma en Annexe~\ref{appdx:schematics}.
\todo{Expliquer brièvement.}

\subsection{Multiplexeur 2 vers 1}
Voir schéma en Annexe~\ref{appdx:schematics}.

\subsubsection{Table de vérité et fonctions logiques}
\todo{Ajouter la table de vérité, $Y$ et $\overline{Y}$.}

\subsubsection{Dimensionnement relatif des transistors}
\todo{Décrire le dimensionnement relatif dans le PUN (pull-up network), le PDN (pull-down) et entre les deux.}

\subsubsection{Dimensionnement absolu}
\todo{Justifier et donner le dimensionnement absolu pour respecter les spécifications.}

\subsection{Tampon de sortie}
Voir schéma en Annexe~\ref{appdx:schematics}.

\subsubsection{Dimensionnement des étages}
\todo{Indiquer le dimensionnement de chaque étage et justifier (schéma ou tableau).}

\section{Mesures}

\subsection{Temps de montée}
\todo{Présenter un tableau avec le temps de montée (10\%-90\%) pour AND, ADD et UAL.}

\subsection{Délai de propagation}
\todo{résenter un tableau avec le délai de propagation pour cin $\rightarrow$ cout, a1 $\rightarrow$ o1 pour AND et ADD.}

\newpage
\appendix



\section{Annexes}

\quicktab{Additionneur 1 Bit}{add1b}{cccccccc}{
  \toprule
  \multirow{2}{*}{$c_{in}$} &
  \multirow{2}{*}{$a$} &
  \multirow{2}{*}{$b$} &
  $T_1$     &
  $D$       &
  $T_2$     &
  $s_{um}$  &
  $c_{out}$ \\
  & & &
  $\overline{(ab)}$ &
  $a\oplus b$ &
  $\overline{(Dc_{in})}$ &
  $D\oplus c_{in}$ &
  $\overline{(T_1T_2)}$
  \\
  0&0&0&1&0&1&0&0 \\
  0&0&1&1&1&1&1&0 \\
  0&1&0&1&1&1&1&0 \\
  0&1&1&0&0&1&0&1 \\
  1&0&0&1&0&1&1&0 \\
  1&0&1&1&1&0&0&1 \\
  1&1&0&1&1&0&0&1 \\
  1&1&1&0&0&1&1&1 \\
  \bottomrule
}


\end{document}
