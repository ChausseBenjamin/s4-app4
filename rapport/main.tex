\documentclass[a11paper]{article}

\usepackage{karnaugh-map}
\usepackage{subcaption}
\usepackage{tabularx}
\usepackage{titlepage}
\usepackage{document}
\usepackage{booktabs}
\usepackage{multicol}
\usepackage{multirow}
\usepackage{float}
\usepackage{longtable}
\usepackage{varwidth}
\usepackage{graphicx}
\usepackage{siunitx}
\usepackage{pifont}
% \usepackage[toc,page]{appendix}
\usepackage[usenames,dvipsnames]{xcolor}

\title{Rapport d'APP}

\class{Logique Combinatoire}
\classnb{GEN420 \& GEN430}

\teacher{Keven Deslandes}

\author{
  \addtolength{\tabcolsep}{-0.4em}
  \begin{tabular}{rcl} % Ajouter des auteurs au besoin
      Benjamin Chausse & -- & CHAB1704 \\
      Shawn Couture    & -- & COUS1912 \\
  \end{tabular}
}

\newcommand{\todo}[1]{\begin{color}{Red}\textbf{TODO:} #1\end{color}}
\newcommand{\note}[1]{\begin{color}{Orange}\textbf{NOTE:} #1\end{color}}
\newcommand{\fixme}[1]{\begin{color}{Fuchsia}\textbf{FIXME:} #1\end{color}}
\newcommand{\question}[1]{\begin{color}{ForestGreen}\textbf{QUESTION:} #1\end{color}}

\newcommand{\quicktab}[4]{
  \begin{table}[H]
    \centering
    \caption{#1}
    \label{tab:#2}
    \begin{tabular}{#3}
      #4
    \end{tabular}
  \end{table}
}

\newcommand{\quickfig}[4]{
\begin{figure}[H]
  \centering
  \includegraphics[width=#3\textwidth]{#4}
  \caption{#1}
  \label{fig:#2}
\end{figure}
}

% Checkboxes
\setlength{\fboxsep}{1pt}
\newcommand{\cbox}{\fbox{\phantom{\ding{51}}}}
\newcommand{\cboxtick}{\fbox{\ding{51}}}%
% self-incrementing Test-ID
\newcounter{tid}
\newcommand{\tid}{\stepcounter{tid}\thetid}

\renewcommand{\frenchtablename}{Tableau}


\begin{document}
\maketitle
\newpage
\tableofcontents
\newpage

\section{Plan de vérification}

\begin{center}
	\begin{longtable}{lp{4.5cm}p{4cm}p{4cm}l}
		% Headers & Footers {{{
		\caption{Plan de vérification} \label{tab:verif}
		\\

		\toprule
		\multicolumn{3}{l}{Objectif Ciblé} &
		\multicolumn{2}{l}{Test des nouvelles opérations}
		\\

		\midrule
		\#                                 &
		\bfseries Test                     &
		\bfseries Action                   &
		\bfseries Résultat Attendu         &
		\cboxtick
		\\

		\midrule
		\endfirsthead

		\multicolumn{5}{c}%
		{{\itshape \tablename\ \thetable{} -- Continué de la page précédente\ldots}}
		\\

		\midrule
		\#                                 &
		Test                               &
		Action                             &
		Résultat Attendu                   &
		\cboxtick
		\\

		\midrule
		\endhead

		\midrule \multicolumn{5}{r}{{Continué à la prochaine page}}
		\\
		\midrule
		\endfoot

		\bottomrule
		\endlastfoot
		% }}}

		% Tests {{{
		\tid                                       &
    Temps de transition Non-Ou &
		Éxecuter \verb|nor_test.asc| et ourvir les logs &
		Le temps de transition \verb|transition_up| et \verb|transition_down| sont
    en bas de 130ps &
		\cbox \\

	\tid                                         &
    Temps de transition Non-Ou                 &
		Éxecuter \verb|nor_test.asc| et ourvir les logs &
		Le temps de transition \verb|transition_up| et \verb|transition_down| sont en bas de 130ps &
		\cbox \\

    % }}}
	\end{longtable}
\end{center}

\section{Conception du Multiplexeur 2 vers 1}

Afin de simplifier la création d'un multiplexeur

\section{Le tampon de sortie}
\todo{Le dimensionnement de chaque étage du tampon de sortie et votre justification (vous
pouvez ajouter par-dessus le schématique ou dans un tableau à part)}

\newpage
\appendix



\section{Annexes}

\quicktab{Additionneur 1 Bit}{add1b}{cccccccc}{
  \toprule
  \multirow{2}{*}{$c_{in}$} &
  \multirow{2}{*}{$a$} &
  \multirow{2}{*}{$b$} &
  $T_1$     &
  $D$       &
  $T_2$     &
  $s_{um}$  &
  $c_{out}$ \\
  & & &
  $\overline{(ab)}$ &
  $a\oplus b$ &
  $\overline{(Dc_{in})}$ &
  $D\oplus c_{in}$ &
  $\overline{(T_1T_2)}$
  \\
  0&0&0&1&0&1&0&0 \\
  0&0&1&1&1&1&1&0 \\
  0&1&0&1&1&1&1&0 \\
  0&1&1&0&0&1&0&1 \\
  1&0&0&1&0&1&1&0 \\
  1&0&1&1&1&0&0&1 \\
  1&1&0&1&1&0&0&1 \\
  1&1&1&0&0&1&1&1 \\
  \bottomrule
}


\end{document}
